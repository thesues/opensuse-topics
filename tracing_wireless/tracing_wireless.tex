\documentclass[adobefonts]{ctexart}
%\documentclass[winfonts]{ctexart}

\CTEXoptions[captiondelimiter={\quad}]


\usepackage{amsmath}            % AMS的数学宏包
\usepackage{amssymb}            % AMS的数学符号宏包
\usepackage{graphicx}           % 插入图片需要的宏包
\usepackage{float}              % 强大的浮动环境控制宏包
\usepackage{framed}             % `shaded'环境需要用到
\usepackage{enumitem}           % 增强列表功能
\usepackage{alltt}              % 在`alltt'环境中为等宽字体, 但可以使用LaTeX命令

% \usepackage{shortvrb}           % 简化\verb的写法
% \MakeShortVerb{\|}
\usepackage{listings}
\lstset{language=bash}
\lstset{extendedchars=false}
\lstset{breaklines}
\lstset{stepnumber=2}
\lstset{backgroundcolor=\color{lightgray}}


\usepackage{color}              % 可以定义各种颜色
\usepackage[x11names]{xcolor}   % 下面的RoyalBlue3颜色需要用到的宏包



% 自定义的几种颜色
\definecolor{shadecolor}{gray}{0.85}

% \definecolor{darkblue}{rgb}{52,101,164}
% \definecolor{darkgreen}{rgb}{78,154,6}

% % 设置背景颜色
% \definecolor{bisque}{rgb}{.996,.891,.755}
% \pagecolor{bisque}

\usepackage[pdfauthor={Dreamseeker},
  colorlinks=true,
  urlcolor=blue,
  linkcolor=RoyalBlue3]{hyperref} % 为超链接设置颜色, 修改PDF文件信息

%\CTEXsetup[name={实验,},number={\chinese{section}}]{section}


% \date{}

\usepackage[pagestyles]{titlesec} % 定制页眉页脚
 % 设置页眉页脚
 \newpagestyle{main}{%
   \sethead[$\cdot$~\thepage~$\cdot$][][\thesection\quad%
   \sectiontitle]{\thesection\quad\sectiontitle}{}{%
   $\cdot$~\thepage~$\cdot$}
   \setfoot{}{\url{http://code.google.com/p/opensuse-topics/}}{}\headrule}
\pagestyle{main}
%\renewpagestyle{plain}{\sethead{}{}{}\setfoot{}{}{}}
%\pagestyle{plain}

\usepackage[top=0.75in,bottom=0.5in,left=1in,right=1in]{geometry} % 设置页边距

\setlength{\belowcaptionskip}{1em} % 设置caption之后的距离

% For LaN
\newcommand{\LaN}{L{\scriptsize\hspace{-0.47em}\raisebox{0.23em}{A}}\hspace{-0.1em}N}
\usepackage{array}
\usepackage{ctable}

\author{BinLi (binli@opensuse.org)}
\title{如何跟踪无线问题}
\begin{document}
\maketitle
\tableofcontents
\newpage
\section{如何跟踪无线问题}
\subsection{引子}
在使用 linux桌面时经常会碰到无法建立无线连接,那么,为了使大家能够快速的解决一些初级的错误,在这里
介绍一下当碰到问题应做些什么来帮助解决问题。当自己无法解决时,那么如何能够给维护人员提供更有用的信
息来减少问题的解决时间。
\subsection{涉及的组件}
建立无线连接主要使用的组件如下:
\begin{itemize}
\item nm-applet (GNOME) 或 KNetworkManager (KDE)
\item NetworkManager 和 YaST
\item wpa\_supplicant
\item 网上驱动
\end{itemize}

\subsection{易犯的错误}
\subsubsection{MAC地址过滤}
注意你的无线接入点(AP)的MAC地址过滤功能,如果已经开启,需要添加你的无线网卡的地址,建议禁用此功能。
\subsubsection{开关}
许多内置的无线网卡可以通过开头来启用或禁用,所以需要注意是否开头已经打开。
\subsubsection{固件}
一些无线网卡驱动需要固件(如iwl3945, iwlagn, b43, ...),许多固件在openSUSE的平台已经提供
(如iwl3945-ucode 和 ralink-firmware),而一些因为版权的原因不能提供,因此您需要自己手动的安装,有
些还可以通过 /usr/bin 下的脚本来安装(如install\_acx100\_firmware,install\_bcm43xx\_firmware 和
install\_intersil\_firmware)。
那么如何决定是否你缺少固件呢,查看dmesg吧
dmesg | less
\subsection{NetworkManager}
当使用 NetworkManager 来管理您的网络的时候碰见问题,以下的日志将会很有用:
\begin{itemize}
\item /var/log/NetworkManager
\item /var/log/wpa\_supplicant.log
\item dmesg 的输出
\end{itemize}

\subsubsection{让 wpa\_supplicant 输出更多的调试信息}
\begin{enumerate}
\item 更改配置文件,系统重启仍有效

  编辑 /usr/share/dbus-1/system-services/fi.epitest.hostap.WPASupplicant.service 文件

  把下面的行

  Exec=/usr/sbin/wpa\_supplicant -c /etc/wpa\_supplicant/wpa\_supplicant.conf -u -f /var/log/wpa\_supplicant.log

  更改为

  Exec=/usr/sbin/wpa\_supplicant -c /etc/wpa\_supplicant/wpa\_supplicant.conf -u -dddt -f /var/log/wpa\_supplicant.log

  用 root 权限来运行下面的命令:
\begin{verbatim}
	> rcnetwork stop
	> killall wpa\_supplicant
	> rcnetwork start
\end{verbatim}

  现在 wpa\_supplicant 将会写更多的日志到 /var/log/wpa\_supplicant.log 文件中。

\item 临时更改

  在程序运行时,openSUSE 提供了一个捕获特定信号来修改调试级别的补丁,这个只是openSUSE 平台我有的,上游的
  代码仍没有这个功能。

  只要运行:
\begin{verbatim}
	> kill -SIGUSR1 `pidof wpa\_supplicant`
\end{verbatim}
  日志文件 /var/log/wpa\_supplicant.log 将会输出更多的调试信息。
\end{enumerate}
\subsection{如何手动建立无线连接}
\begin{enumerate}
\item 停止 NetworkManager 和 wpa\_supplicant
\begin{verbatim}
	> rcnetwork stop
	> killall wpa\_supplicant
\end{verbatim}
  如果无线接入点没使用安全设置,或 WEP 模式的安全链接,使用 iwconfig 就可以建立连接。
  但所有其它的连接(WPA-PSK, WPA-EAP, 802.1x 动态 WEP )都需要使用wpa\_supplicant。

\item 基本命令

  首先用 iwconfig 来找到无线网卡所使用的接口。
\begin{verbatim}
	> iwconfig
	lo        no wireless extensions.

	eth1      no wireless extensions.

	eth0      unassociated  ESSID:""
	Mode:Managed  Frequency=2.412 GHz  Access Point: Not-Associated
	Bit Rate:0 kb/s   Tx-Power=20 dBm   Sensitivity=8/0
	Retry limit:7   RTS thr:off   Fragment thr:off
	Encryption key:off
	Power Management:off
	Link Quality:0  Signal level:0  Noise level:0
	Rx invalid nwid:0  Rx invalid crypt:3  Rx invalid frag:0
	Tx excessive retries:0  Invalid misc:53   Missed beacon:0
\end{verbatim}

  在这个例子中 eth0 是无线网上的接口。

  如果您不知道无线接入点的名字(Essid),可以使用下面命令扫描。
\begin{verbatim}
	> iwlist eth0 scan
	...
	Cell 02 - Address: XX:XX:XX:XX:XX:XX
	ESSID:"XXXXXX"
	Protocol:IEEE 802.11bg
	Mode:Master
	Channel:2
Frequency:2.417 GHz (Channel 2)
	Encryption key:off
	Bit Rates:1 Mb/s; 2 Mb/s; 5.5 Mb/s; 6 Mb/s; 9 Mb/s
	11 Mb/s; 12 Mb/s; 18 Mb/s; 24 Mb/s; 36 Mb/s
	48 Mb/s; 54 Mb/s
	Quality=37/100  Signal level=-78 dBm
	Extra: Last beacon: 488ms ago
	...
\end{verbatim}

  如果您的无线接入点并没有显示出来,可能是由于没有广播essid,那么可以扫描特定的名字,如下:

\begin{verbatim}
	> iwlist eth0 scan essid your\_essid\_here
	...
\end{verbatim}
\end{enumerate}

\subsection{借助监听模式抓包}
\subsection{引用链接}

\end{document}

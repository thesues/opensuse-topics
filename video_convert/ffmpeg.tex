\documentclass[adobefonts]{ctexart}
%\documentclass[winfonts]{ctexart}

\CTEXoptions[captiondelimiter={\quad}]


\usepackage{amsmath}            % AMS的数学宏包
\usepackage{amssymb}            % AMS的数学符号宏包
\usepackage{graphicx}           % 插入图片需要的宏包
\usepackage{float}              % 强大的浮动环境控制宏包
\usepackage{framed}             % `shaded'环境需要用到
\usepackage{enumitem}           % 增强列表功能
\usepackage{alltt}              % 在`alltt'环境中为等宽字体, 但可以使用LaTeX命令

% \usepackage{shortvrb}           % 简化\verb的写法
% \MakeShortVerb{\|}
\usepackage{listings}
\lstset{language=bash}
\lstset{extendedchars=false}
\lstset{breaklines}
\lstset{stepnumber=2}
\lstset{backgroundcolor=\color{lightgray}}


\usepackage{color}              % 可以定义各种颜色
\usepackage[x11names]{xcolor}   % 下面的RoyalBlue3颜色需要用到的宏包
% 自定义的几种颜色
\definecolor{shadecolor}{gray}{0.85}

% \definecolor{darkblue}{rgb}{52,101,164}
% \definecolor{darkgreen}{rgb}{78,154,6}

% % 设置背景颜色
% \definecolor{bisque}{rgb}{.996,.891,.755}
% \pagecolor{bisque}

\usepackage[pdfauthor={Dreamseeker},
  pdftitle={用ffmpeg转换视频格式},
  colorlinks=true,
  urlcolor=blue,
  linkcolor=RoyalBlue3]{hyperref} % 为超链接设置颜色, 修改PDF文件信息

%\CTEXsetup[name={实验,},number={\chinese{section}}]{section}


\title{\textbf{用ffmpeg转换视频格式}}
\author{deanraccoon@gmail.com}
% \date{}

\usepackage[pagestyles]{titlesec} % 定制页眉页脚
% % 设置页眉页脚
% \newpagestyle{main}{%
%   \sethead[$\cdot$~\thepage~$\cdot$][][\thesection\quad%
%   \sectiontitle]{\thesection\quad\sectiontitle}{}{%
%   $\cdot$~\thepage~$\cdot$}
%   \setfoot{}{}{}\headrule}
% \pagestyle{main}
% \renewpagestyle{plain}{\sethead{}{}{}\setfoot{}{}{}}
\pagestyle{plain}

\usepackage[top=0.75in,bottom=0.5in,left=1in,right=1in]{geometry} % 设置页边距

\setlength{\belowcaptionskip}{1em} % 设置caption之后的距离

% For LaN
\newcommand{\LaN}{L{\scriptsize\hspace{-0.47em}\raisebox{0.23em}{A}}\hspace{-0.1em}N}
\begin{document}

\maketitle
\tableofcontents
\newpage
\section{实用fffmpeg命令}
\begin{verbatim}
转自
http://www.catswhocode.com/blog/19-ffmpeg-commands-for-all-needs
	转自
	http://www.jcartier.net/spip.php?article36
\end{verbatim}
\subsection{查看视频文件信息} 
\begin{verbatim}
ffmpeg -i video.avi
\end{verbatim}
\subsection{N个图片转化视频}
\begin{verbatim}
ffmpeg -f image2 -i image%d.jpg video.mpg
\end{verbatim}
这个命令把当前文件夹下所有图片转成视频文件

\subsection{视频转化图片}
\begin{verbatim}
ffmpeg video.mpg image%d.jpg
\end{verbatim}
图片格式包括:PGM, PPM, PAM, PGMYUV, JPEG, GIF, PNG, TIFF, SGI
\subsection{视频转为iphone可播放的格式}
\begin{verbatim}
ffmpeg source_video.avi input -acodec aac -ab 128kb -vcodec mpeg4 -b 1200kb -mbd 2 -flags 
+4mv+trell -aic 2 -cmp 2 -subcmp 2 -s 320x180 -title X final_video.mp4
\end{verbatim}
参数含义:
\begin{itemize}
\item 源文件: source\_video.avi
\item 音频编码 : aac
\item 音频的比特率 : 128kb/s
\item 视频编码 : mpeg4
\item 视频比特率 : 1200kb/s
\item 视频大小 : 320px par 180px
\item 结果文件 : final\_video.mp4
\end{itemize}

\subsection{视频转换成psp格式}
\begin{verbatim}
ffmpeg -i source_video.avi -b 300 -s 320x240 -vcodec xvid -ab 32 -ar 24000 -acodec aac final_video.mp4
\end{verbatim}
参数含义:
\begin{itemize}
\item 源文件: source\_video.avi
\item 音频编码 : aac
\item 音频比特率 : 32kb/s
\item 视频编码 : xvid
\item 视频比特率 : 1200kb/s
\item 视频大小: 320px par 180px
\item 结果文件: final\_video.mp4
\end{itemize}

\subsection{提取视频音轨}
\begin{verbatim}
ffmpeg source_video.avi -vn -ar 44100 -ac 2 -ab 192 -f mp3 sound.mp3
\end{verbatim}
参数含义:
\begin{itemize}
\item 源文件 : source\_video.avi
\item 音频比特率 : 192kb/s
\item 输出格式 : mp3
\item 结果文件 : sound.mp3
\end{itemize}

\subsection{wav转成mp3}
\begin{verbatim}
ffmpeg -i son_origine.avi -vn -ar 44100 -ac 2 -ab 192 -f mp3 son_final.mp3
\end{verbatim}

\subsection{avi转mpg}
\begin{verbatim}
ffmpeg video_origine.avi video_finale.mpg
\end{verbatim}

\subsection{mpg转avi}
\begin{verbatim}
ffmpeg video_origine.mpg video_finale.avi
\end{verbatim}

\subsection{avi转gif(无压缩)}
\begin{verbatim}
ffmpeg video_origine.avi gif_anime.gif
\end{verbatim}

\subsection{混合视频和音频}
\begin{verbatim}
ffmpeg -i son.wav -i video_origine.avi video_finale.mpg
\end{verbatim}

\subsection{avi转flv}
\begin{verbatim}
ffmpeg video_origine.avi -ab 56 -ar 44100 -b 200 -r 15 -s 320x240 -f flv video_finale.flv
\end{verbatim}

\subsection{avi转dv}
\begin{verbatim}
ffmpeg video_origine.avi -s pal -r pal -aspect 4:3 -ar 48000 -ac 2 video_finale.dv
或者:
ffmpeg -i video_origine.avi -target pal-dv video_finale.dv
\end{verbatim}

\subsection{avi转dvd players}
\begin{verbatim}
ffmpeg -i source_video.avi -target pal-dvd -ps 2000000000 -aspect 16:9 finale_video.mpeg
\end{verbatim}
参数解释:
\begin{itemize}
\item target pal-dvd : 输出文件格式
\item ps 2000000000: 限制输出文件的占用最大空间(单位bits,这里是表示2G比特,换算成字节是250M字节)
\item aspect 16:9 : 宽高比 16:9 表示是宽屏幕,如果是普通屏幕,是4:3
\end{itemize}

\subsection{avi转divx}
\begin{verbatim}
ffmpeg video_origine.avi -s 320x240 -vcodec msmpeg4v2 video_finale.avi
\end{verbatim}

\subsection{ogm转mpeg dvd}
\begin{verbatim}
ffmpep film_sortie_cinelerra.ogm -s 720x576 -vcodec mpeg2video -acodec mp3 film_termin.mpg
\end{verbatim}

\subsection{avi转SVCD mpeg2}
\begin{verbatim}
ffmpeg -i video_origine.avi -target ntsc-svcd video_finale.mpg
\end{verbatim}

\subsection{avi转VCD mpeg2}
NTSC制式(日本,美国)
\begin{verbatim}
ffmpeg -i video_origine.avi -target ntsc-vcd video_finale.mpg
\end{verbatim}

PAL制式(欧洲,中国)
\begin{verbatim}
ffmpeg -i video_origine.avi -target pal-vcd video_finale.mpg
\end{verbatim}
\input{../readme}
\end{document}
